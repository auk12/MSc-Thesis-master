\section{Discussion}

\subsection{Principal Results}
\paragraph{The Hear phase revealed three general areas in which the CHVs were involved in maternal and child health care: home visits with pregnant women and new mothers, monitoring new deliveries in the community, and emergency care. Within each of these areas, three key themes emerged as priorities for the CHVs in designing a new patient management system: the need for fast, easy, data reporting methods, improved communication between CHVs and the clinic regarding referrals, and the need for an effective, reliable way to report and respond to obstetric emergencies. These themes led to the inclusion of five key design features that complemented the daily tasks of CHVs in the study catchment area, which are outlined as follows:}
\begin{enumerate}
	\item Reporting of home visits through IVR
	\item Notification of referred patients visiting the clinic through text message
	\item Reporting of deliveries through IVR
	\item Notification of patients delivering at the clinic through text message
	\item Reporting of emergencies directly to the clinic through IVR
\end{enumerate}

\paragraph{Mock testing of these features during the Create phase revealed enthusiasm and satisfaction with the service, but concerns were raised regarding the ability for CHVs to flash the Baby Monitor phone number should they run out of phone credit. This issue was addressed by including m-Pesa payments, which can be transferred directly into phone credit, for CHVs participating in the study.}

\paragraph{A total of 1,312 calls were made from CHV phone numbers over the course of the study period. However, not all of these calls were made with the intention of reporting a home visit, delivery, or emergency. Some of these calls may have been made by CHVs seeking more information about the system or simply calling to test or get acquainted with using the system. Some of these calls may have been initiated, but dropped either due to user error (i.e. hanging up the call) or poor network coverage.}

\paragraph{A total of 401 'valid' calls \textemdash made with the intention of reporting a deliver a home visit, delivery, or emergency \textemdash from CHVs were made during the study period. On average, this meant that each of the 55 CHVs triggered an average of seven calls into the system over the study  period. 95 home visits to the 175 enrolled pregnant women were reported by CHVs, while 71 of the 72 deliveries by enrolled women during the study period were reported by CHVs. Although the number of home visits to enrolled women indicates that not all enrolled women were visited by CHVs during the study period, the system was able to capture almost all of the deliveries by women enrolled with Baby Monitor over the course of the study period. Thus, while there is room to improve with respect to encouraging more frequent home visits, results showed that such a patient management system could be used to successfully report deliveries in a timely fashion.}

\paragraph{As shown in Table~\ref{tab:usagevsreport}, the number of home visits and deliveries reported during the study period was considerably lower than the data reported by CHVs through biweekly reports in the six months prior to implementation of the system. From January through June of 2013,  approximately 29 deliveries per month were reported by CHVs in this catchment area. In the eight month study period that followed from July 2013 to March 2014, approximately 9 deliveries per month were reported by CHVs through the patient management system. This difference can be attributed to inconsistent functioning of the system during the first four months of implementation. After launch of the system, numerous problems were encountered with the telecommunications company that allowed the transmission of calls from the Verboice platform to users' mobile phones. As a result, CHVs were unable to use the system at numerous points during this window, accounting for the decline in call volume during the first four months of the study. The increase in call volume observed over the last four months of the study coincided with the resolution of the aforementioned problems.}

\begin{table}[h]
  \centering
  \caption[Comparison of CHV report data from biweekly reports vs. patient management system reports]{On a per month basis, number of home visits and births reported during the study period were lower compared to the data reported by the same group of CHVs through biweekly reports in the six months prior to the launch of the system.}
    \begin{tabular}{rrr}
    \toprule
    \textbf{Indicator} & \textit{No. per month (1/2013-6/2013)} & \textit{No. per month (7/2013-3/2014)} \\
    \midrule
    Homes visited & 236   & 12 \\
    Births & 29    & 9 \\
    \bottomrule
    \end{tabular}%
  \label{tab:usagevsreport}%
\end{table}%

\paragraph{Despite fluctuations in call volume during the study period, the CHVs found the system to be highly usable. With 96\% of users responding to the usability survey, results showed that users generally believed that the system was easy to use, useful for their daily tasks, easy to interact with, and improved their quality of work life. This indicated that users believed that the system, for the most part, met the needs and challenges identified by the CHV focus group during the Hear phase of the study. The text message components of the system were rated especially high by respondents, indicating that communication via text may be an area worth pursuing in future iterations of the study.}

\subsection{Limitations}
\paragraph{This study piloted an intervention that was previously untested in this target population. Thus, the scope of the study was limited to a single clinic and a small, convenience sample of CHVs working in the clinic's catchment area. Participants for focus groups were selected based on English comprehension and demonstrated interest in the study, and thus may not generally represent the perspectives or viewpoints of all other CHVs within the health system. Due to time constraints, only one cycle of the Hear and Create phases were completed; in future iterations, additional cycles of prototyping and mock testing would have been conducted in order to refine and create additional features for the system.} 

%\subsection{Comparison with Prior Work}
%
%- MoTECH program: IVR/texts in Ghana for midwives
%- Believe that this has potential to improve community-based maternal and child health care

\subsection{Conclusions}
\paragraph{Maternal and child mortality remains a public health problem in Kenya. Although many mHealth programs have implemented worldwide to improve maternal and child health care, few have leveraged both interactive voice response and text messaging modalities to engage with both patients and providers. This pilot study adopted a human-centered design framework to identify and address the challenges faced by community health volunteers in providing maternal and child health services. Based on these findings, a patient management system was created that allowed CHVs to report data and receive text message updates about patients in their communities. This system was generally well-received by the CHVs, as they used the system with regularity and provided high self-reported usability ratings. }
