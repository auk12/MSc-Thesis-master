\section{Methods}
\subsection{Project Setting}
\paragraph{Kenya is divided into 47 counties, each of which is comprised of a number of divisions. Each of these divisions is subdivided into numerous individual community units. The current health system is divided into four tiers that reflect these geographic divisions: national referral hospitals, county hospitals, primary care services, and community level services \citep{SPA2010}.}

\paragraph{Health care delivery at the level of the community unit is the foundation of this system. According to the Community Strategy guidelines published by the Kenyan Ministry of Health in 2006, community level care should be focused on prevention of infectious diseases, control of noncommunicable diseases, maternal and child health services, family planning activities, and effective sanitation and hygiene practices \citep{CommunityStrategy2006}. Providers at this level are the community  health volunteers (CHVs), who opt to serve their communities on a voluntary basis. Identified and selected by their fellow community members, CHVs are the primary interface between individuals and the public health system.}

\paragraph{According to the Ministry of Health guidelines released in 2006, each CHV should be responsible for 20 households or 100 individuals \citep{CommunityStrategy2006}. Community health extension workers (CHEWs) are responsible for training and supervising a cadre of 25 CHVs each. In a single community unit, 2 CHEWs should supervise approximately 50 CHVs in order to provide care to 5,000 people \citep{CommunityStrategy2006}. Typically, CHEWs are based out of the local dispensary, collecting information from CHVs and visiting their assigned community units on a regular basis.}

\paragraph{Dispensaries and clinics serve clusters of community units. These facilities aim to provide both preventative and curative services, including prenatal care, family planning, and basic emergency care. Some of these facilities may also be equipped to handle deliveries. Providers at this level of care include nurses, and clinical officers.}

\paragraph{This study was conducted in the Ndivisi Division of Bungoma County, a rural region approximately 60km west of Eldoret. Ndivisi, which has a population of 77,599, is divided into 11 community units \citep{Census2009}. The research site was comprised of two community units, Sinoko and Sitabicha, with a combined population of 10,744 \citep{Census2009}. These two community units were local to Sinoko clinic, the largest clinic in Ndivisi division. This clinic is one of only three health facilities in the area with the personnel, equipment, and supplies to handle deliveries on a regular basis. The two remaining facilities are located in Webuye town, which is approximately 16km away from the clinic.}

\paragraph{At the time of the study, there were 55 CHVs and 3 CHEWs working in the study catchment area. On average, each CHV was responsible for approximately 195 individuals and 36 households \textemdash  far more than the 100 individuals and 20 households suggested by the Ministry of Health's Community Strategy guidelines. Most individual villages in the study community units only had one CHV assigned to provide services.}


\subsection{Recruitment}
\paragraph{For nurses and CHVs to participate in the study, they were required to be comfortable speaking in both English and Swahili and comfortable using a mobile phone to receive calls and text messages.}

\paragraph{At the time of recruitment, the staff at the clinic included one clinical officer, who served as the head administrator, and four nurses. 55 CHVs also reported to the clinic at least once per month to provide information on the families living in their villages within the clinic catchment area. Of these providers, three nurses and six CHVs, each representing a different village, were selected to participate based on the inclusion criteria and interest in the project. Upon selection, verbal and written informed consent was obtained from the nurses and CHVs prior to study participation.}

\paragraph{CHVs were asked to recruit the population of pregnant women for the study. To be eligible, CHVs were asked to identify women who were pregnant, between the ages of 18 and 35, reside within Ndivisi Dviison, have access to a mobile phone, and be able to speak either English or Swahili. If a CHV identified a woman who met each of the inclusion criteria, they were asked to encourage or accompany women to visit the clinic to enroll in the study. If an eligible woman visited the clinic, the participating nurses were asked to record the woman's name, up to three different phone numbers for the woman and her household,  community unit and village of residence, expected delivery date, gravidity, and other basic demographic information including age, marital status, and education level. This information was stored in the Baby Monitor database and was used to connect enrolled women to the CHVs responsible for providing care in their village.}

\subsection{Design Framework}
\paragraph{The overall objective of this study was to develop a system that could fill existing gaps and fit within the current health infrastructure in Kenya. Ideally, this system would leverage text and voice interactions to better connect patients, CHVs, and clinic nurses so as to improve care-seeking behaviors and overall delivery of maternal and child health care. In order to build such a system, the research team adopted a human-centered design framework. Within this framework, the users of the system are targeted from the beginning of the research process. Throughout the design and development phases, the users are regularly consulted in order to ensure that the end product meets their unique needs and priorities. Methods for this study were adopted from the Human-Centered Design Toolkit, a collection of strategies and techniques focused on developing solutions that meet the needs of users in the developing world \citep{HCDToolkit}. This toolkit defined three iterative phases of the human-centered design process: Hear, Create, and Deliver. }

\subsection{Hear Phase}
\paragraph{The methods employed during this phase included a focus group discussion with CHVs, shadow days with CHVs, and a focus group discussion with the clinic nurses. The focus group discussion with the CHVs was loosely structured, with the research team asking a series of open-ended questions regarding CHV roles, responsibilities, and work flow related to maternal and child health. CHVs were asked to expand on topics such as data collection and patient referral within the discussion as well.}

\paragraph{Following this focus group discussion, the research team shadowed two of the CHVs participating in the focus group on two separate occasions. This allowed for a better understanding of the CHVs' daily responsibilities and experiences conducting home visits within their village. It also gave the CHVs an opportunity to describe some of the challenges that they face in visiting homes, collecting information, and managing care for the entirety of their village.}

\paragraph{The final aspect of the Hear phase was a focus group discussion with nurses staffed at the study clinic. Similarly, the discussion was facilitated by the research team and focused on the nurses' experiences working with pregnant women and new mothers with emphasis on patient referral. This allowed for a better understanding of the referral process from the clinic side, and gave the nurses a chance to voice their concerns, frustrations, and suggestions for improving the methods by which CHVs and nurses convey information to one another. Using these findings, the research team was able to identify a set of themes and design principles that would govern the development process going forward.}


\subsection{Create Phase}

\paragraph{The prototype system integrated several technologies: Verboice, a platform for designing and initiating phone calls over the internet, a Voice over Internet Protocol (VoIP) provider in Kenya, a software framework called Asterisk that connected Verboice to the VoIP provider, a telecommunications company in Kenya that delivered the calls to the mobile phones of users, and an SMS gateway provider that sent text messages to the users' mobile phones. An analysis engine, written in R, integrated each of these technologies to trigger new calls through Verboice, trigger text messages through the SMS gateway provider, and process call data. This analysis engine was responsible for ensuring interoperability between each of the individual components, allowing information to be collected from through IVR and be transmitted back to CHVs via text message.}

\subsubsection{Verboice}
\paragraph{The system was designed in Verboice, an open source platform for creating projects that interact with end-users via voice and text, and R, an open source statistical computing environment. Verboice allows end-users to listen to audio messages in multiple languages, respond to questions with the phone keypad, and  record their own voice messages. Using the web-based Verboice platform, the research team built upon the existing Baby Monitor platform to create call flows designed for use by CHVs at the clinic. Each call flow was designed with the same basic framework. First, the user calls into the Baby Monitor system and immediately hangs up \textemdash  a process known as `''flashing'' a number. This is a common practice in Kenya, especially when a mobile phone user does not wish to be charged for an incoming call. After the user flashes the Baby Monitor number, the user receives a free incoming call through Verboice. During this call, the user listens series of instructions, questions, and prompts that require numeric input from the user's phone keypad, and was designed to address the design principles and themes identified for the patient management system during the Hear phase. For questions that required a 'yes' or 'no' answer, users were asked to press '1' or '3' on their keypads. For other questions, users were also asked to enter numerical data through their keypads. No data or answers to questions were stored locally on their phones; all responses to all questions were saved to the research team's Verboice database.}

\subsubsection{SMS Gateway}
\paragraph{The research team also created a set of text messages specific to the roles and responsibilities of the CHVs in order to supplement the interactive voice response system. These messages were designed to use information provided by the CHVs in previous calls with the system to help them complete their daily responsibilities. These messages were automated by the analysis script in R and were delivered to users by the local SMS gateway provider.}

\subsubsection{Mock Testing}
\paragraph{In order to test these call flows and automated text messages, the research team conducted a mock testing session with the CHV focus group. Index cards with text were used to represent each audio or text message, and volunteers were selected to read the messages aloud to the group. This was done in order to confirm the content and logical flow of the messages and questions, and to gain feedback on the strengths and weaknesses of the system. Based on feedback from this focus group session, the research team finalized the content and flow of each message in the call flow within the web-based Verboice platform. A woman native to Ndivisi and familiar with the local dialects was recruited to assist in translation of all messages and recording of the audio messages in English and Swahili. Recording was completed at a studio in a nearby town.}


\subsection{Deliver Phase}
\paragraph{The three nurses previously selected to participate in the study and the full sample of 55 CHVs were chosen to pilot the patient management system with patients within the clinic catchment area. Prior to launch of the system, meetings were held with the full group of CHVs and the group of nurses to brief them on their responsibilities during the study. The primary outcomes for this evaluation phase were (1) usage of the system by CHVs and (2) self-reported usability rating of the system.}

\paragraph{Data regarding the usage of the patient management system  was collected over the course of eight months. This length of time was selected in order to allow CHVs who identified newly pregnant women at the onset of the study to follow these patients throughout the course of their pregnancies. Number of calls initiated by CHVs, number of home visits reported, and number of deliveries reported were collected during this time period. This information was collected and stored on the web-based Verboice platform.}

\paragraph{Usability testing began approximately six months after launch of the system. A modified version of the Health IT Usability Evaluation Scale \citep{Yen2010} was administered to all CHVs through a Verboice call flow (see Table~\ref{tab:usabilitysurvey}). Participants were called through Verboice via an automated R script and listened to a series of statements regarding the perceived usefulness, perceived ease of use, user interface, and quality of work life. Using their numeric keypads, they were asked to press '1' to agree with the statement and '3' to disagree. They were subsequently asked to whether they agreed or disagreed 'a lot' or 'a little'. As a modified form of a Likert scale, this instrument allowed CHVs to quantify the system's overall usability.}



