\section{Introduction}
\paragraph{Every day, approximately 800 women die of complications related to pregnancy, childbirth, or abortion around the world. According to the World Health Organization (WHO), an estimated 287,000 maternal deaths occcurred in 2010 alone \citep{WHO2012}. Most maternal deaths occur between the third trimester and the first six weeks after delivery \textemdash   the majority of which occur either during or within the first few days after delivery \citep{WHO2012}. The most common causes of maternal death during this period are severe bleeding, hypertensive disease, and infection \citep{WHO2012}. Moreover, the burden of maternal mortality is greatest among developing countries where most low-income women deliver in their own homes. In sub-Saharan Africa, one in every sixteen women will die of pregnancy-related causes \textemdash   a lifetime risk higher than anywhere else in the world \citep{Ronsmans2006}.}

\paragraph{Most maternal deaths are avoidable. At least 80\% of maternal deaths can be prevented by a set of proven interventions provided by a skilled practitioner. Two-thirds of all infant deaths can be prevented with postnatal care provided by a health practitioner during the first six weeks after birth. However, delays in recognizing the need to seek care, accessing health care facilities, and receiving  adequate care make the delivery of the aforementioned interventions extremely challenging \citep{Thaddeus1994}. }

\paragraph{These three delays have disproportionately affected women and families living in rural and remote regions. High costs of care, far distances between villages and health facilities, and relative lack of skilled practitioners in these regions all contribute to lower prenatal care coverage and deliveries at health facilities among women in rural areas. Thus, pregnancy-related complications such as anemia or infection, which would normally be identified early on during the course of a pregnancy, commonly go unnoticed. These complications, when not addressed, can prove to be deadly for both mother and child during the critical days during and after delivery.}

\paragraph{Community-based interventions have emerged as potentially effective methods in reducing the delays associated with maternal mortality. Since gaps exist between the community and the health facility, whether due to costs, distance, or other factors, the common theme among community-based strategies has been to extend provision of care into villages and individual households. In recent years, some promising community-based interventions for maternal and child health have focused on using mobile phones.}

\subsection{mHealth}
\paragraph{Over the past decade, mobile phones have had an incredible impact on low to middle income countries. Mobile phone technology has enabled millions of people to communicate to and from some of the most poor and remote areas of the world \textemdash  especially in sub-Saharan Africa \citep{Adler2007}. Moreover, increased phone penetration has allowed mobile providers to expand the roles of mobile phones  beyond that of simple communication devices.}

\paragraph{In 2007, Safaricom \textemdash  the largest mobile provider in Kenya \textemdash  launched a mobile phone-based payment service called m-Pesa. Designed for the ''unbanked'', m-Pesa allowed users to make deposits and withdrawals, transfer and receive money to and from others, pay bills, and purchase airtime through a simple interface accessible on all mobile phones. As of 2010, m-Pesa has been adopted by 9 million users, roughly 40\% of Kenya's adult population \citep{Mas2010}. This model of mobile banking has been replicated in a number of developing countries, including Uganda, Tanzania, and India.}
 
\paragraph{The success of m-Pesa and other mobile payment systems set a precedent for the use of mobile phone technology in developing countries. As mobile phone penetration has continued to increase, mobile phone technology has been applied in a variety of contexts in the health care space. These applications have largely aimed to address gaps and challenges that exist within health systems in developing countries \citep{Labrique2013}. The earliest of these interventions involved using mobile phones as a primary method of data collection, allowing health workers to report data immediately at the point of care. This strategy has been used to implement mobile-phone based vital registration systems (such as Uganda Mobile VRS) and establish electronic health record systems(such as OpenMRS), both of which rely on data entry at the point of care and allow for data collection in rural or remote areas \citep{Labrique2013}.}

\subsection{mHealth in Maternal and Child Health}
\paragraph{Over the past decade, mHealth technologies have been widely implemented in the field of maternal and child health. Specifically, mHealth programs have been used to expand data collections to reach financially and geographically isolated populations, provide support and information for providers at the point of care, improve response to obstetric emergencies, and promote healthy behaviors among pregnant women and new mothers \citep{Tamrat2012}.}

\paragraph{Many of these interventions have used text-based interactions as the primary mode of interacting with both providers and patients. For example, text messages have been used to train and educate midwives about safe delivery and postnatal care practices in South Africa \citep{Woods2012}. Text message reminders have also been used to improve timeliness of routine visits by community health workers in Tanzania \citep{DeRenzi2012}. The Mobile Alliance for Maternal Action (MAMA) has created a package of text messages that provide educational information to pregnant women and new mothers throughout their pregnancies and one year post-delivery \citep{MAMA}. Interventions centered around MAMA messages have been implemented in several developing countries, including South Africa, India, and Bangladesh. In each of these countries, MAMA messages were adapted for each region based onthe known cultural norms and beliefs regarding pregnancy and child care \citep{McCartney2012}. These programs may also help improve the overall patient experience for pregnant women who have opted to receive prenatal care. Studies have shown that pregnant women who received biweekly text messages offering support during the time between prenatal care visits had higher satisfaction levels with their care than women who did not receive any messages \citep{Jareethum2008}.}

\paragraph{Compared to text message-based interventions, relatively few mHealth programs have focused on using voice-based interaction. These programs have primarily focused on using voice-based applications to engage with community level providers, rather than patients. The Obstetric Helpline program in Rajastan, India has enabled community members and health workers to connect patients to the appropriate health facilities during emergencies, thereby attempting to reduce the delays associated with seeking and receiving care \citep{UNICEF2008}.  The Healthline Project, a speech-based IVR system currently in development in Pakistan, has attempted to improve access to information for community health workers at the point of care \citep{Sherwani2007}. }

\paragraph{The Mobile Technology for Community Health (MOTECH) program in Ghana is one unique mHealth initiative that has implemented both IVR and text message interventions in the field of maternal and child health. The MOTECH program is comprised of two components: one targeted at pregnant women, the other targeted at community health workers and nurses. MOTECH uses IVR and text messaging as options for communicating with and educating patients, while using text messages as a way to send alerts or reminders for follow up care to community health workers \citep{MOTECH2011}. While this program is still in development, results from the pilot phase of implementation suggested that most users preferred to interact in their native language via IVR, rather than receiving text message with reminders or educational information \citep{MOTECH2011}.}

\subsection{Baby Monitor}
\paragraph{Although the established literature has provided examples of various programs with promising elements, there is a need for more mHealth programs that integrate voice and text interfaces to engage with both patients and providers in the maternal and child health space. In 2012, principal investigator Eric Green and his research team at the Population Council began development and testing of a new mHealth service called Baby Monitor. In the pilot phase of this project, the Baby Monitor team partnered with InSTEDD, a non-profit technology group, and Jacaranda Health, a non-profit maternity clinic in Nairobi, to develop and refine a health screening system that reaches pregnant women directly through their mobile phones via IVR. Participants completed automated screening calls and identical, follow-up clinical screenings with a nurse at Jacaranda Health at several points before and after delivery. Calls were scheduled based on the WHO guidelines for focused prenatal care and the Kenyan Ministry of Health's guidelines for postnatal immunizations. The results of this pilot phase have yet to be published, but the mobile screens were found to be reliable when compared to the in person follow-up assessments. Moreover, uptake for the service was high and women reported that they enjoyed receiving calls from the Baby Monitor system.}

\paragraph{This project was built upon the existing Baby Monitor framework, and aimed to develop a comprehensive voice and text-based system that would complement the patient-centered screening service. This system would engage with providers in order to act on the results provided by the screening service; that is to track patient referrals, monitor women throughout the stages of their pregnancies, and avoid delays in seeking care, reaching facilities, and receiving care. The intent of this pilot study was to design and test a preliminary system that would help community health volunteers (CHVs), the primary providers of maternal and child health in Kenya, manage referrals and track the progress of pregnant women living in their communities. This project will ultimately inform future research focused on using phone-based screening results to provide decision making support for CHVs in identifying and targeting women at a high risk for pregnancy-related complications.}
