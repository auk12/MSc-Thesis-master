\section{Abstract}
\textsc{Background:} Most maternal deaths are avoidable. Delays in recognizing the need to seek care, delays in accessing  health care facilities, and delays in receiving adequate care can all make delivery of effective maternal health care practices very difficult. In recent years, mobile phones have grown in popularity for improving disease prevention and management, especially in the field of maternal and child health. \\
\textsc{Objective:} The intent of this study was to design and pilot a mobile-phone based patient management system intended for use by community health volunteers, who serve as the primary providers of maternal and child health care in Kenya.  \\
\textsc{Methods:}  Using a human-centered design framework, community health volunteers and clinic nurses helped shape the system and evaluated the pilot system for usability.\\
\textsc{Results:} The patient management system was found to be highly usable, with 94\% of respondents agreeing with the notion that the system helped them do their jobs better. \\
\textsc{Keywords:} maternal health, infant health, mHealth, patient referral, health informatics
