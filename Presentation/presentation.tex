%%%%%%%%%%%%%%%%%%%%%%%%%%%%%%%%%%%%%%%%%%%%%%%%%%%%%%%%%%%%
%%  This Beamer template was created by Cameron Bracken.
%%  Anyone can freely use or modify it for any purpose
%%  without attribution.
%%
%%  Last Modified: January 9, 2009
%%

\documentclass[xcolor=x11names,compress]{beamer}

%% General document %%%%%%%%%%%%%%%%%%%%%%%%%%%%%%%%%%

\usepackage{graphicx}
\usepackage{tikz}
\usetikzlibrary{decorations.fractals}
%%%%%%%%%%%%%%%%%%%%%%%%%%%%%%%%%%%%%%%%%%%%%%%%%%%%%%
\graphicspath{{./Pictures/}}

%% Beamer Layout %%%%%%%%%%%%%%%%%%%%%%%%%%%%%%%%%%
\useoutertheme[subsection=false,shadow]{miniframes}
\useinnertheme{default}
\setbeamertemplate{navigation symbols}{}
\setbeamertemplate{blocks}[rounded][shadow=false]


\setbeamertemplate{footline}
{
\begin{beamercolorbox}[wd=\paperwidth,ht=0.3ex,dp=1.125ex,%
	 leftskip=.3cm,rightskip=.3cm plus1fil]{upper separation line foot}
\end{beamercolorbox} %

\begin{beamercolorbox}[wd=\paperwidth,ht=2.5ex,dp=1.125ex,%
	 leftskip=.3cm,rightskip=.3cm plus1fil]{footlinecolor}%
        \usebeamerfont{section in head/foot}%
	\color{gray} \insertshortauthor\hfill\insertframenumber{} of \inserttotalframenumber{}

    \end{beamercolorbox}

}



\usefonttheme{serif} % default family is serif
\usepackage{fontspec}
\defaultfontfeatures{Numbers=OldStyle}
\setmainfont{Contra}


\setbeamerfont{title like}{shape=\scshape}
\setbeamerfont{frametitle}{shape=\scshape}


\setbeamercolor*{lower separation line head}{bg=DodgerBlue3}
\setbeamercolor*{upper separation line foot}{bg=DodgerBlue3}

\setbeamercolor*{footlinecolor}{fg=black,bg=black!10}
\setbeamercolor*{normal text}{fg=black,bg=white} 
\setbeamercolor*{alerted text}{fg=Firebrick3} 
\setbeamercolor*{example text}{fg=DarkOliveGreen1} 
\setbeamercolor*{structure}{fg=black} 
\setbeamercolor*{palette tertiary}{fg=black,bg=black!10} 
\setbeamercolor*{palette quaternary}{fg=black,bg=black!10} 

\renewcommand{\(}{\begin{columns}}
\renewcommand{\)}{\end{columns}}
\newcommand{\<}[1]{\begin{column}{#1}}
\renewcommand{\>}{\end{column}}



%%%%%%%%%%%%%%%%%%%%%%%%%%%%%%%%%%%%%%%%%%%%%%%%%%




\begin{document}

\title{Design and Usability Testing of a Mobile Phone-Based Patient Management System for Women in Rural Kenya}
\author[Amogh Karnik]{
	Amogh Karnik\\
	{\it M.Sc. Candidate}\\
}
\institute[DGHI]{\includegraphics[scale=0.15]{dghi}}
\date{\today}
%%%%%%%%%%%%%%%%%%%%%%%%%%%%%%%%%%%%%%%%%%%%%%%%%%%%%%
%%%%%%%%%%%%%%%%%%%%%%%%%%%%%%%%%%%%%%%%%%%%%%%%%%%%%%

\begin{frame}
\titlepage
\end{frame}

\begin{frame}{Overview}
\tableofcontents
\end{frame}


%%%%%%%%%%%%%%%%%%%%%%%%%%%%%%%%%%%%%%%%%%%%%%%%%%%%%%
%%%%%%%%%%%%%%%%%%%%%%%%%%%%%%%%%%%%%%%%%%%%%%%%%%%%%%
\section{Introduction}
\subsection{Maternal Mortality}

\begin{frame}{What we know...}

\begin{block}{
Reducing maternal mortality is a major global health priority. }
%Every day, 800 women die of pregnancy-related complications around the world.
\end{block}
\begin{block}{
Most maternal deaths take place during a specific time period.}
%Most of these deaths take place during or within the first few days after childbirth.
\end{block}
\begin{block}{
The burden of maternal mortality is greatest in poor and remote areas.}
%Compared to women worldwide, 1 of 16 women in sub-Saharan Africa will die of pregnancy-related complications.
\end{block}
\end{frame}


\begin{frame}{What we know...}

\uncover<1->{\begin{block}{Most maternal deaths are avoidable.}
%Skilled practices can prevent at least 80% of maternal deaths and 67% of all newborn deaths. 
\end{block}}

\uncover<2->{\begin{alertblock}{Three delay model for maternal mortality:}
\begin{enumerate}
\item<3->{Seeking care}
\item<4->{Accessing care}
\item<5->{Receiving care}
\end{enumerate}

\end{alertblock}}

%Explain three delays. These delays make safe maternal and child health practices very difficult to deliver. 
\end{frame}

\subsection{mHealth}
\begin{frame}[t]{Mobile Phones and mHealth}
\begin{columns}[T]
\column{2in}
\begin{itemize}
	\item<1->{m-Pesa:}
		\begin{itemize}
		\item<1->{Mobile banking for everyone}
		\end{itemize}
	\item<2->{Magpi, OpenDataKit, Formhub:}
		\begin{itemize}
		\item<2->{Mobile data collection at the point of care}
		\end{itemize}
\end{itemize}
\column{2in}
\begin{itemize}
	\item<3->{Text message interventions}
		\begin{itemize}
		\item<3->{Patient education, health promotion}
		\item<3->{Provider training}
		\end{itemize}
	\item<4->{Interactive voice response (IVR)}
		\begin{itemize}
		\item<4->{Patient education}
		\item<4->{Emergency response}
		\end{itemize}
\end{itemize}
\end{columns}
\end{frame}

\begin{frame}{Baby Monitor}

\end{frame}





%%%%%%%%%%%%%%%%%%%%%%%%%%%%%%%%%%%%%%%%%%%%%%%%%%%%%%
%%%%%%%%%%%%%%%%%%%%%%%%%%%%%%%%%%%%%%%%%%%%%%%%%%%%%%
\section{Methods}
\subsection{Setting}
\begin{frame}{Kenyan Health System}
\only<1>{ }
\centerline{\includegraphics<2>[scale=0.42]{community}
\includegraphics<3>[scale=0.42]{clinic}
\includegraphics<4>[scale=0.42]{county}
\includegraphics<5>[scale=0.42]{health-system}}
\end{frame}

\begin{frame}{Research Site}
\begin{columns}[c]
\column{2in}
\includegraphics[width=2in]<1->{map}
\column{2in}
\begin{itemize}
	\item<2->{Two community units}
	\item<2->{Population: 10,744}
	\item<3->{Clinic equipped for deliveries}
	\item<4->{55 CHVs}
		\begin{itemize}
		\item<5->{195 individuals}
		\item<5->{36 households}
		\end{itemize}
	\end{itemize}

\end{columns}
\end{frame}


\subsection{Human-Centered Design}
\begin{frame}{Human-Centered Design}
\only<1>{
\begin{figure}
\centerline{\includegraphics[scale=0.35]<1>{HCD}}
\end{figure}}
\end{frame}

\begin{frame}[t]{Hear Phase}
\begin{block}{\textcolor{DodgerBlue4}{Objective: to understand the users, their responsibilities, needs, and environment.}}
\uncover<2->{How does the current system of community-based maternal and child health care work?
\begin{itemize}
\item{CHV focus group discussion}
\item{CHV shadow days}
\item{Clinic nurse focus group discussion}
\end{itemize}}
\end{block}
\end{frame}

\begin{frame}[t]{Create Phase}
\begin{block}{\textcolor{DodgerBlue4}{Objective: to develop a design solution based on what we've ''heard''.}}
\uncover<2->{How can voice and text interfaces be integrated to address the users' stated needs and specifications?
\begin{itemize}
\item{Verboice}
\item{VoIP, Asterisk, telecommunications company}
\item{SMS gateway provider}
\item{Analysis engine in R}
\item{CHV mock testing}
\end{itemize}}
\end{block}
\end{frame}

\begin{frame}{Deliver Phase}
\begin{block}{\textcolor{DodgerBlue4}{Objective: to implement and evaluate the design solution.}}
\uncover<2->{How well did the design solution address the users' stated needs and specifications?
\begin{itemize}
\item{Usage: call data from July 2013 - March 2014}
\item{Usability: evaluation survey administered through an automated Verboice call}
\begin{itemize}
\item{Perceived ease of use}
\item{Perceived usefulness}
\item{User control}
\item{Quality of work life}
\end{itemize}
\end{itemize}}
\end{block}
\end{frame}


%%%%%%%%%%%%%%%%%%%%%%%%%%%%%%%%%%%%%%%%%%%%%%%%%%%%%%
%%%%%%%%%%%%%%%%%%%%%%%%%%%%%%%%%%%%%%%%%%%%%%%%%%%%%%
\section{Results}
\subsection{System Design}
\begin{frame}{Create Phase}
These are the results of the create phase.
\end{frame}

\subsection{Usage and Usability}
\begin{frame}{Deliver Phase}

\end{frame}

%%%%%%%%%%%%%%%%%%%%%%%%%%%%%%%%%%%%%%%%%%%%%%%%%%%%%%
%%%%%%%%%%%%%%%%%%%%%%%%%%%%%%%%%%%%%%%%%%%%%%%%%%%%%%
\section{Discussion}
\subsection{Frame 1}
\begin{frame}{Frame 1}

\end{frame}



%%%%%%%%%%%%%%%%%%%%%%%%%%%%%%%%%%%%%%%%%%%%%%%%%%%%%%
%%%%%%%%%%%%%%%%%%%%%%%%%%%%%%%%%%%%%%%%%%%%%%%%%%%%%%
\end{document}