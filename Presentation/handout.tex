%%%%%%%%%%%%%%%%%%%%%%%%%%%%%%%%%%%%%%%%%%%%%%%%%%%%%%%%%%%%
%%  This Beamer template was created by Cameron Bracken.
%%  Anyone can freely use or modify it for any purpose
%%  without attribution.
%%
%%  Last Modified: January 9, 2009
%%

%%%%%%% SLIDE SHOW %%%%%%%%%%%%%%%%
\documentclass[xcolor=x11names, handout, compress]{beamer}
\usepackage{pgfpages}
\pgfpagesuselayout{2 on 1}[a4paper,border shrink=5mm]


%% General document %%%%%%%%%%%%%%%%%%%%%%%%%%%%%%%%%%

\usepackage{graphicx}
\usepackage{tikz}
\usepackage{media9}
%%%%%%%%%%%%%%%%%%%%%%%%%%%%%%%%%%%%%%%%%%%%%%%%%%%%%%
\graphicspath{{./Pictures/}}

%% Beamer Layout %%%%%%%%%%%%%%%%%%%%%%%%%%%%%%%%%%
\useoutertheme[subsection=false,shadow]{miniframes}
\useinnertheme{default}
\setbeamertemplate{navigation symbols}{}
\setbeamertemplate{blocks}[rounded][shadow=false]


\setbeamertemplate{footline}
{
\begin{beamercolorbox}[wd=\paperwidth,ht=0.3ex,dp=1.125ex,%
	 leftskip=.3cm,rightskip=.3cm plus1fil]{upper separation line foot}
\end{beamercolorbox} %

\begin{beamercolorbox}[wd=\paperwidth,ht=2.5ex,dp=1.125ex,%
	 leftskip=.3cm,rightskip=.3cm plus1fil]{footlinecolor}%
        \usebeamerfont{section in head/foot}%
	\color{gray} \insertshortauthor\hfill\insertframenumber{} of \inserttotalframenumber{}

    \end{beamercolorbox}

}



\usefonttheme{serif} % default family is serif
\usepackage{fontspec}
\defaultfontfeatures{Numbers=OldStyle}
\setmainfont{Contra}


\setbeamerfont{title like}{shape=\scshape}
\setbeamerfont{frametitle}{shape=\scshape}


\setbeamercolor*{lower separation line head}{bg=DodgerBlue3}
\setbeamercolor*{upper separation line foot}{bg=DodgerBlue3}

\setbeamercolor*{footlinecolor}{fg=black,bg=black!10}
\setbeamercolor*{normal text}{fg=black,bg=white} 
\setbeamercolor*{alerted text}{fg=Firebrick3} 
\setbeamercolor*{example text}{fg=DarkOliveGreen1} 
\setbeamercolor*{structure}{fg=black} 
\setbeamercolor*{palette tertiary}{fg=black,bg=black!10} 
\setbeamercolor*{palette quaternary}{fg=black,bg=black!10} 

\renewcommand{\(}{\begin{columns}}
\renewcommand{\)}{\end{columns}}
\newcommand{\<}[1]{\begin{column}{#1}}
\renewcommand{\>}{\end{column}}

\AtBeginSection[]
{
\begin{frame}
\frametitle{Overview}
\tableofcontents[currentsection]
\end{frame}
}

%%%%%%%%%%%%%%%%%%%%%%%%%%%%%%%%%%%%%%%%%%%%%%%%%%




\begin{document}


\title{Design and Usability Testing of a Mobile Phone-Based Patient Management System for Women in Rural Kenya}
\author[Amogh Karnik]{
	Amogh Karnik\\
	{\it M.Sc. Candidate}\\
}
\institute[DGHI]{\includegraphics[scale=0.15]{dghi}}
\date{\today}
%%%%%%%%%%%%%%%%%%%%%%%%%%%%%%%%%%%%%%%%%%%%%%%%%%%%%%
%%%%%%%%%%%%%%%%%%%%%%%%%%%%%%%%%%%%%%%%%%%%%%%%%%%%%%

\begin{frame}
\titlepage
\end{frame}

\begin{frame}{Overview}
\tableofcontents
\end{frame}


%%%%%%%%%%%%%%%%%%%%%%%%%%%%%%%%%%%%%%%%%%%%%%%%%%%%%%
%%%%%%%%%%%%%%%%%%%%%%%%%%%%%%%%%%%%%%%%%%%%%%%%%%%%%%
\section{Introduction}
\subsection{Maternal Mortality}

\begin{frame}{What we know...}

\begin{block}{
Reducing maternal mortality is a major global health priority. }
%Every day, 800 women die of pregnancy-related complications around the world.
\end{block}
\begin{block}{
Most maternal deaths take place during a specific time period.}
%Most of these deaths take place during or within the first few days after childbirth.
\end{block}
\begin{block}{
The burden of maternal mortality is greatest in poor and remote areas.}
%Compared to women worldwide, 1 of 16 women in sub-Saharan Africa will die of pregnancy-related complications.
\end{block}
\end{frame}


\begin{frame}{What we know...}

\uncover<1->{\begin{block}{Most maternal deaths are avoidable.}
%Skilled practices can prevent at least 80% of maternal deaths and 67% of all newborn deaths. 
\end{block}}

\uncover<2->{\begin{alertblock}{Three delay model for maternal mortality:}
\begin{enumerate}
\item{Seeking care}
\item{Accessing care}
\item{Receiving care}
\end{enumerate}

%Mobile phones have emerged in the last decade as a way of addressing each of these delays as part of community-based strategies and interventions. 
\end{alertblock}}
\end{frame}

\subsection{mHealth}
\begin{frame}[t]{Mobile Phones and mHealth}
\begin{columns}[T]
\column{2in}
\begin{itemize}
	\item<1->{m-Pesa:}
		\begin{itemize}
		\item<1->{Mobile banking for everyone}
		\end{itemize}
	\item<2->{Magpi, OpenDataKit, Formhub:}
		\begin{itemize}
		\item<2->{Mobile data collection at the point of care}
		\end{itemize}
\end{itemize}
\column{2in}
\begin{itemize}
	\item<3->{Text message interventions}
		\begin{itemize}
		\item<3->{Patient education, health promotion}
		\item<3->{Provider training}
		\end{itemize}
	\item<4->{Interactive voice response (IVR)}
		\begin{itemize}
		\item<4->{Patient education}
		\item<4->{Emergency response}
		\end{itemize}
\end{itemize}
\end{columns}
\end{frame}

\begin{frame}{Baby Monitor}
\begin{itemize}
\item{Targets pregnant women directly with IVR}
\item{Women answer screening questions by pressing numbers on their keypads}
\item{Pilot study in Nairobi showed that screenings were reliable compared to in-person assessments with nurses}
\item{Second study conducted in parallel to this project: assess reliability and validity in a rural, remote population}
\end{itemize}
\end{frame}

\begin{frame}{Research Objectives}
\begin{itemize}
\item{To understand the roles of CHVs, their responsibilities, needs, and environment}
\item{To design a patient management system that addresses these characteristics}
\item{To implement and evaluate the design solution based on feedback from the CHVs}
\end{itemize}
\end{frame}

%%%%%%%%%%%%%%%%%%%%%%%%%%%%%%%%%%%%%%%%%%%%%%%%%%%%%%
%%%%%%%%%%%%%%%%%%%%%%%%%%%%%%%%%%%%%%%%%%%%%%%%%%%%%%
\section{Methods}
\subsection{Setting}
\begin{frame}{The Health System in Kenya}
%Describe geographic divisions, counties, community units, villages.
\centerline{\includegraphics<1|handout:0>[scale=0.42]{community}
\includegraphics<2|handout:0>[scale=0.42]{clinic}
\includegraphics<3|handout:0>[scale=0.42]{county}
\includegraphics<4|handout:1>[scale=0.42]{health-system}}
\end{frame}

\begin{frame}{Maternal Health Care in Kenya}
\begin{itemize}
\item{Primary delivered at community level}
\item{\textbf{Free} at all public health facilities as of June 1, 2013}
\item{CHV responsibilities:}
\begin{itemize}
\item{Pre- and post-natal home visits}
\item{Identify and monitor women throughout pregnancy}
\item{Family planning services}
\item{\textbf{Maternal and child health services}}
\end{itemize}
\end{itemize}
\end{frame}

\begin{frame}{Research Site}
\begin{columns}[c]
	\column{2in}
	\begin{itemize}
			\item{Two community units}
			\item{Population: 10,744}
			\item{Clinic equipped for deliveries}
			\item{55 CHVs}
				\begin{itemize}
					\item{195 individuals}
					\item{36 households}
				\end{itemize}
	\end{itemize}
	\column{2.5in}
	\includegraphics[width=2.25in]{map_portrait}

\end{columns}
\end{frame}


\subsection{Human-Centered Design}
\begin{frame}[t]{Human-Centered Design}
\centerline{\includegraphics[scale=0.17]{hcd}}
\end{frame}

\begin{frame}[t]{Hear Phase}
\begin{block}{\textcolor{DodgerBlue4}{Objective: to understand the users, their responsibilities, needs, and environment.}}
\uncover<2->{How does the current system of community-based maternal and child health care work?
\begin{itemize}
\item{CHV focus group discussion}
\item{CHV shadow days}
\item{Clinic nurse focus group discussion}
\end{itemize}}
\end{block}
\end{frame}

\begin{frame}[t]{Create Phase}
\begin{block}{\textcolor{DodgerBlue4}{Objective: to develop a design solution based on what we've ''heard''.}}
\uncover<2->{How can voice and text interfaces be integrated to address the users' stated needs and specifications?
\begin{itemize}
\item{Verboice}
\item{VoIP, Asterisk, telecommunications company}
\item{SMS gateway provider}
\item{Analysis engine in R}
\item{CHV mock testing}
\end{itemize}}
\end{block}
\end{frame}

\begin{frame}[t]{Deliver Phase}
\begin{block}{\textcolor{DodgerBlue4}{Objective: to implement and evaluate the design solution.}}
\uncover<2->{How well did the design solution address the users' stated needs and specifications?
\begin{itemize}
\item{Usage: call data from July 2013 - March 2014}
\item{Usability: evaluation survey administered through an automated Verboice call}
\end{itemize}}
\end{block}
\end{frame}


%%%%%%%%%%%%%%%%%%%%%%%%%%%%%%%%%%%%%%%%%%%%%%%%%%%%%%
%%%%%%%%%%%%%%%%%%%%%%%%%%%%%%%%%%%%%%%%%%%%%%%%%%%%%%
\section{Results}
\subsection{System Design}
\begin{frame}[t]{Reporting Data}
\begin{itemize}
\item{CHVs submit reports every two weeks to the clinic}
\item{Approximately 5-6 months to visit each household in each village}
\item{Home visit information is hand-written, paper based}
\item{Collecting data on number of deliveries in the community is a key component of reports}
\item{Nurses rarely used CHV reports; presents challenges for preparing for prenatal, postnatal care at the clinic}
\end{itemize}
\end{frame}

\begin{frame}[t]{Reporting Data}
\begin{block}{\textcolor{IndianRed2}{Design Principle: Reporting home visits through IVR}}
\uncover<2->{\begin{itemize}
\item{CHV ''flashes'' the Baby Monitor number, receives free call back}
\item{Identify themselves as CHVs with their national ID number \only<2|handout:0>{\includemedia[
  addresource=audio1.mp3,
  flashvars={
    source=audio1.mp3
   &autoPlay=true
  }
]{(example)}{APlayer.swf}
}
\only<3|handout:0>{\includemedia[
  addresource=audio2.mp3,
  flashvars={
    source=audio2.mp3
   &autoPlay=true
  }
]{(example)}{APlayer.swf}
}}
\end{itemize}}


\uncover<2->{
\centerline{\includegraphics<2|handout:1>[scale=0.30]{rephome1}
\includegraphics<3|handout:2>[scale=0.30]{rephome2}}}
\end{block}
\end{frame}

\begin{frame}[t]{Reporting Data}
\begin{block}{\textcolor{IndianRed2}{Design Principle: Reporting deliveries through IVR}}
\uncover<2->{\begin{itemize}
\item{CHV ''flashes'' the Baby Monitor number, receives free call back}
\item{Identify themselves as CHVs with their national ID number}
\end{itemize}}

\uncover<2->{
\centerline{\includegraphics<2|handout:1>[scale=0.30]{repdel11}
\includegraphics<3|handout:2>[scale=0.30]{repdel22}}}
\end{block}
\end{frame}



\begin{frame}[t]{Patient Referral}
\begin{itemize}
\item{CHVs carry ''referral books'' with sheets given to patients to take to clinic}
\item{Nurses estimated that 50 patients per week referred by CHVs}
\item{CHVs have no way of knowing whether patients followed up on their referrals}
\item{CHVs have no way of hearing about deliveries if not contacted directly}

\end{itemize}
\end{frame}


\begin{frame}[t]{Patient Referral}
%Clinic visit notification
\begin{block}{\textcolor{IndianRed2}{Design Principle: Referral notifications through text message}}
\uncover<2->{\begin{itemize}
\item{Visits from enrolled pregnant women logged by clinic nurses, data entered into Baby Monitor database}
\item{R analysis script matched each woman who visited the clinic to the CHV assigned to her village of residence}
\item{Automated text messages sent the following morning}
\end{itemize}}
\uncover<2->{\textit{Hi. Betty Odong visited the clinic yesterday! This was her ANC 2 month visit. Please encourage her to continue attending appointments.}}
\end{block}
\end{frame}

\begin{frame}[t]{Patient Referral}
%Delivery notification
\begin{block}{\textcolor{IndianRed2}{Design Principle: Delivery notifications through text message}}
\uncover<2->{\begin{itemize}
\item{Family member ''flashes'' Baby Monitor number, receives free call back}
\item{Identical to CHV reporting call flow}
\item{R analysis script matches the woman reported to the CHV assigned to her village}
\item{Automated text messages sent the following morning}
\end{itemize}}
\uncover<2->{\textit{Hi. Betty Odong delivered her baby on 08-04-2014!}}
\end{block}
\end{frame}

\begin{frame}[t]{Emergency Response}
\begin{itemize}
\item{CHVs are usually called during an emergency}
\item{Recommend that the patient go to the clinic for immediate care}
\item{Often, the clinic was unprepared to handle an emergency case}
\item{Little to no direct communication between CHVs and clinic nurses about incoming emergencies}
\end{itemize}
\end{frame}

\begin{frame}[t]{Emergency Response}
%Emergency Reporting
\begin{block}{\textcolor{IndianRed2}{Design Principle: Reporting emergencies through IVR}}
\uncover<2->{\begin{itemize}
\item{CHV, mother, or family member ''flashes'' the Baby Monitor number, receives free call back}
\item{Indicate that they would like to report an emergency}
\end{itemize}}
\uncover<2->{\centerline{\includegraphics<2|handout:1>[scale=0.25]{reportem2}}}
\end{block}
\end{frame}


\subsection{Usage and Usability}
\begin{frame}{Call Volume}
\begin{itemize}
\item{1,312 total calls registered from CHVs}
\item{401 valid calls registered from CHVs}
\item{Call volume fluctuated over the eight month period}
\item{CHVs reported 95 home visits and 71 deliveries during this period}
\end{itemize}
\end{frame}

\begin{frame}{Usability Results}
\centerline{\includegraphics[width=4.7in]{Picture3}}
\end{frame}

%%%%%%%%%%%%%%%%%%%%%%%%%%%%%%%%%%%%%%%%%%%%%%%%%%%%%%
%%%%%%%%%%%%%%%%%%%%%%%%%%%%%%%%%%%%%%%%%%%%%%%%%%%%%%
\section{Discussion}
\subsection{Lessons Learned}
\begin{frame}{Lessons Learned}
\begin{itemize}
\item{Oral translation of messages}
\item{Quality of voice messages}
\item{Mobile network variability}
\end{itemize}
\end{frame}

\begin{frame}{Limitations}
\begin{itemize}
\item{Pilot study: small convenience sample}
\item{Time constraints: one single iteration of HCD cycle}
\end{itemize}
\end{frame}

\subsection{Future Research}
\begin{frame}{Future Research}
\begin{itemize}
\item{Impact on process outcomes: home visits, clinic visits for prenatal and postnatal care, deliveries}
\item{Integration of screening service: decision-making support for CHVs}
\item{Additional features suggested by focus group participants: reminders for upcoming events}
\item{Considerations for scaling up:}
\begin{itemize}
\item{Long-term cost of IVR}
\item{Patient enrollment strategies}
\item{CHV engagement strategies}
\end{itemize}
\end{itemize}
\end{frame}


%%%%%%%%%%%%%%%%%%%%%%%%%%%%%%%%%%%%%%%%%%%%%%%%%%%%%%
%%%%%%%%%%%%%%%%%%%%%%%%%%%%%%%%%%%%%%%%%%%%%%%%%%%%%%
\end{document}