\documentclass[pdf]{beamer}
\mode<all>{\usetheme{Warsaw} \useoutertheme{splitRSI}}
\mode<handout>{\usecolortheme{seagull}}
\usepackage{rsislidepacks}
\usepackage{colortbl}
\usepackage{epstopdf}
\usenavigationsymbolstemplate{}

% title information
\title{Making Slides}
\subtitle{...and doing it with Beamer}
\author{RSI \the\year\ Staff}

\begin{document}

\AtBeginSection[]{
\begin{frame}{Table of Contents}
	\tableofcontents[currentsection]
\end{frame}
}

\begin{frame}
	\thispagestyle{empty}
	\titlepage
\end{frame}
\addtocounter{framenumber}{-1}

\begin{frame}{Table of Contents}
	\tableofcontents
\end{frame}

%%%%%%%%%%%%%%%%%%%%%%%%%%%%%%%%%%%%%%%%
\section{Intro to Beamer}

%%%%%%%%%%%%%%%%%%%%%%%%
\subsection{About Beamer}

\begin{frame}{What Is Beamer?}
	\begin{itemize}
		\item Beamer is a flexible \LaTeX\ class for making slides and presentations.
		\item It supports functionality for making PDF slides complete with colors, overlays, environments, themes, transitions, etc.
		\item Adds a couple new features to the commands you've been working with.
		\pause
		\item As you probably guessed, this presentation was made using the Beamer class.
	\end{itemize}
\end{frame}

%%%%%%%%%%%%%%%%%%%%%%%%
\subsection[Basic Structure]{Basic Structure}

\begin{frame}[fragile]{Document Template: slides.tex}
\begin{columns}[t]
  \column{0.5\textwidth}
  \begin{block}{}
    \begin{verbatim}
\documentclass[pdf]
          {beamer}
\mode<presentation>{}
%% preamble
\title{The title}
\subtitle{The subtitle}
\author{your name}

\begin{document}
  	\end{verbatim}	
  \end{block}
  
	\column{0.5\textwidth}
	\begin{block}{}
		\begin{semiverbatim}
\%\% title frame
\\begin\{frame\}
    \\titlepage
\\end\{frame\}

\%\% normal frame
\\begin\{frame\}\{Frame title\}
    The body of the frame.
\\end\{frame\}

\\end\{document\}
		\end{semiverbatim}
	\end{block}
\end{columns}
\verb+athena% make slides.pdf+
\end{frame}

\begin{frame}[fragile]{What would you like in your sandwich?}
\begin{itemize}
	\item So what can you do between \verb=\begin{frame}= and \verb=\end{frame}=?
	\pause
	\item Pretty much anything you can do in a normal \LaTeX\
	document:
	\pause
	\begin{itemize}
		\item figures, tables, equations, normal text, etc.
	\end{itemize}
\end{itemize}
\end{frame}

%\begin{frame}[fragile]{Too much \LaTeX\ for your frames to handle?}
%	\begin{itemize}
%		\item Be careful not to fit too much text, especially when writing mathematical formulas, but if you must...
%		\pause
%		\item Use the optional argument \verb=[allowframebreaks]= with \verb=\begin{frame}= to let \LaTeX\ automatically segment your very long frame.
%		\pause
%		\item Warning: Overlays are not available with \verb=[allowframebreaks]=.
%	\end{itemize}
%\end{frame}

\begin{frame}{Don't Do This}
\begin{itemize}
	\item Here is a well-known formula:\vspace{-.5em}
	$$\displaystyle \sum_{k=0}^{n} k = \frac{n(n+1)}{2}$$
	\item Here is a less well-known, but still useful, formula:\vspace{-.5em}
	$$\displaystyle \sum_{k=0}^{n} k^2 = \frac{n(n+1)(2n+1)}{6}$$
	\item This is pretty well-known, too:\vspace{-.5em}
	$$\displaystyle \sum_{k=0}^{n} k^3 = \left(\frac{n(n+1)}{2}\right)^2$$
	\item Who knows about this one?\vspace{-.5em}
	$$\displaystyle \sum_{k=0}^{n} k^4 = \frac{n(6n^4 + 15n^3 + 10n^2 + 1)}{30}$$
	\item Have fun factoring the quartic expression!
\end{itemize}
\end{frame}

%%%%%%%%%%%%%%%%%%%%%%%%%%%%%%%%%%%%%%%%
\section{Overlaying Concepts}

\begin{frame}{The Rudimentary pause}
Watch this slide grow.
\pause
\begin{itemize}
	\item Hello, World!
	\pause
	\item Hello, Mars!
	\pause
	\item Hello, Alpha Centauri!
\end{itemize}
\end{frame}

\begin{frame}[fragile]{The Rudimentary pause: Backstage}
\begin{semiverbatim}
Watch this slide grow.
\alert{\\pause}
\\begin\{itemize\}
  \\item Hello, World!
  \alert{\\pause}
  \\item Hello, Mars!
  \alert{\\pause}
  \\item Hello, Alpha Centauri!
\\end\{itemize\}
\end{semiverbatim}
\end{frame}

%%%%%%%%%%%%%%%%%%%%%%%%
\subsection{Specifications}

\begin{frame}[fragile]{The Specification}
\begin{itemize}
\item<+-| alert@1-> Professor: I want you to read the textbook to prepare for tomorrow's lecture.
\item<+-> Student: Which chapter should I read?
\item<+-| alert@3-> Professor: \textit{Specifically}, Chapters \LARGE{\verb=<1-3, 6, 10->=}.
\end{itemize}
\end{frame}

\begin{frame}{Specificationizing the Rudimentary pause}
Watch this slide grow.
\begin{itemize}
	\item<2-> Hello, World!
	\item<3-> Hello, Mars!
	\item<4-> Hello, Alpha Centauri!
\end{itemize}
\end{frame}

\begin{frame}[fragile]{Specificationizing the Rudimentary pause: Backstage}
\begin{semiverbatim}
Watch this slide grow.
\\begin\{itemize\}
  \alert{\\item<2->} Hello, World!
  \alert{\\item<3->} Hello, Mars!
  \alert{\\item<4->} Hello, Alpha Centauri!
\\end\{itemize\}
\end{semiverbatim}
\end{frame}

\begin{frame}[fragile]{Useful Commands that Work with Specifications}
\begin{footnotesize}
\begin{table}
\begin{tabular}{|p{.75in}|p{1in}|p{.75in}|p{1in}|} \hline
  \verb=\textbf<>{}= & controls when to bold text
 & \verb=\only<>{}= & controls when to reveal text, occupies NO space otherwise \\ \hline
  \verb=\textit<>{}= & controls when to italicize text
 & \verb=\uncover<>{}= & controls when to reveal text, DOES occupy space
 otherwise \\ \hline
  \verb=\color<>[]{}= & controls when to change color of text
 & \verb=\alt<>{}{}= & reveals first argument when specification is
 true, otherwise reveals second argument \\ \hline
  \verb=\alert<>{}= & controls when to highlight text (default red)
 & \verb=\item<>= & controls when an item is shown \\ \hline
\end{tabular}
\end{table}
\end{footnotesize}
\end{frame}

%%%%%%%%%%%%%%%%%%%%%%%%
\subsection[Examples]{Examples: Lists, Graphics, Tables}

\begin{frame}{Lists: The \$1,000,000 Question}
Which president said, ``Most folks are about as happy as they make up their minds to be''?
\begin{enumerate}[A]
\item<2-5> James Madison
\item<3-5> Harry Truman
\item<4-> \color<6>[rgb]{0,0.6,0}Abraham Lincoln
\item<5-5> Calvin Coolidge
\end{enumerate}
\uncover<1-5>{Hints:}\\
\uncover<2-5>{\small{James Madison ate broccoli.}}\\
\uncover<3-5>{\small{Harry Truman drank milk.}}\\
\uncover<4-5>{\small{Abe Lincoln raised bees.}}\\
\uncover<5-5>{\small{And Cal Coolidge grew silk.}}\\
\end{frame}

\begin{frame}[fragile]{Lists: The \$1,000,000 Question: Backstage}
\begin{verbatim}
\begin{enumerate}[A]
    \item<2-5> James Madison
    \item<3-5> Harry Truman
    \item<4-> \color<6>[rgb]{0,0.6,0}Abraham Lincoln
    \item<5-5>  Calvin Coolidge
\end{enumerate}

\uncover<1-5>{Hints:}\\
\uncover<2-5>{James Madison ate broccoli.}\\
\uncover<3-5>{Harry Truman drank milk.}\\
\uncover<4-5>{Abe Lincoln raised bees.}\\
\uncover<5-5>{And Cal Coolidge grew silk.}\\
\end{verbatim}
\end{frame}

\begin{frame}{Columns and Blocks}
\begin{columns}
  \column{0.5\textwidth}
  \begin{figure}[ht]
  \begin{center}
  \includegraphics[height=2in]{LarsonGifted.eps}~\footnote{Apologies to
  Gary Larson}
  \end{center}
  \end{figure}
  
  \column{0.5\textwidth}
  \begin{block}<2->{Observation 1}
    Simmons Hall is composed of metal and concrete.
  \end{block}
  \begin{block}<3->{Observation 2}
    Simmons Dormitory is composed of brick.
  \end{block}
  \begin{block}<4->{Conclusion}
    Simmons Hall $\not=$ Simmons Dormitory.
  \end{block}
\end{columns}
\end{frame}

\begin{frame}[fragile, allowframebreaks=0.8]{Columns and Blocks: Backstage}
\begin{verbatim}
\begin{columns}
  \column{0.5\textwidth}
    \begin{figure}[ht]
    \begin{center}
        \includegraphics[height=2in]{LarsonGifted.eps}
            ~\footnote{Apologies to Gary Larson}
    \end{center}
    \end{figure}
  



  \column{0.5\textwidth}
      \begin{block}<2->{Observation 1}
          Simmons Hall is composed of metal and concrete.
      \end{block}
      \begin{block}<3->{Observation 2}
          Simmons Dormitory is composed of brick.
      \end{block}
      \begin{block}<4->{Conclusion}
          Simmons Hall $\not=$ Simmons Dormitory.
      \end{block}
\end{columns}
\end{verbatim}
\end{frame}

%%%%%%%%%%%%%%%%%%%%%%%%%%%%%%%%%
\newtheorem{thm}{Easy Theorem}
\newtheorem{pf}{Proof}
\newtheorem{rmk}{Remark}
%%%%%%%%%%%%%%%%%%%%%%%%%%%%%%%%%
\begin{frame}{Math stuff}
	\begin{thm}<1->
	The equation
	$$x^n+y^n=z^n,$$
	has no integer solutions for $n>2$ where $x,\,y,\,z\,\neq\,0$.
	\end{thm}
	\begin{pf}<3->
	The proof is trivial and left as an exercise for the reader.
	\end{pf}
	\begin{rmk}<2->
	This problem was first posed in $10,000$ B.C.
	\end{rmk}
\end{frame}

\begin{frame}[fragile]{Math stuff: Backstage}
\begin{small}
\begin{verbatim}
\newtheorem{thm}{Easy Theorem}
\newtheorem{pf}{Proof}
\newtheorem{rmk}{Remark}

\begin{thm}<1->
    $$x^n+y^n=z^n,$$
    has no integer solutions for $n>2$ 
    where $x,\,y,\,z\,\neq\,0$.
\end{thm}
\begin{pf}<3->
    The proof is trivial and left as an exercise.
\end{pf}
\begin{rmk}<2->
    This problem was first posed in $10,000$ B.C.
\end{rmk}
\end{verbatim}
\end{small}
\end{frame}

\begin{frame}{Building Tables}
\begin{table}[bt]
\begin{tabular}{|l|c|c|} \hline
\textbf{Ice Cream Store} & \textbf{Location} & \textbf{How to Get There} \\ \hline
\uncover<2->{Toscanini's} & \uncover<2->{Central Square} & \uncover<2->{Just walk!} \\
\uncover<3->{Herrell's} & \uncover<3->{Harvard Square} & \uncover<3->{Red Line} \\
\uncover<4->{J.P. Licks} & \uncover<4->{Davis Square} & \uncover<4->{Red Line} \\
\uncover<5->{Ben \& Jerry's} & \uncover<5->{Newbury Street} & \uncover<5->{Green Line} \\ \hline
\end{tabular}
\end{table}

%\rowcolors[]{1}{blue!20}{blue!10}
%\begin{tabular}{l|cccc}
%Class & A & B & C & D \\\hline
%X & 1 & 2 & 3 & 4 \pause\\
%Y & 3 & 4 & 5 & 6 \pause\\
%Z & 5 & 6 & 7 & 8
%\end{tabular}

\end{frame}

\begin{frame}[fragile]{Building Tables: Backstage}
\begin{scriptsize}
\begin{verbatim}
\begin{table}[bt]
\begin{tabular}{|l|c|c|} \hline
  \textbf{Ice Cream Store}      & \textbf{Location} 
                                & \textbf{How to Get There}   \\ \hline
  \uncover<2->{Toscanini's}     & \uncover<2->{Central Square} 
                                & \uncover<2->{Just walk!}    \\
  \uncover<3->{Herrell's}       & \uncover<3->{Harvard Square} 
                                & \uncover<3->{Red Line}      \\
  \uncover<4->{J.P. Licks}      & \uncover<4->{Davis Square} 
                                & \uncover<4->{Red Line}      \\
  \uncover<5->{Ben \& Jerry's}  & \uncover<5->{Newbury Street} 
                                & \uncover<5->{Green Line}    \\ \hline
\end{tabular}
\end{table}
\end{verbatim}
\end{scriptsize}
\end{frame}
%%%%%%%%%%%%%%%%%%%%%%%%%%%%%%%%%%%%%%%%
\section[Sparkle]{Adding that Sparkle}

%%%%%%%%%%%%%%%%%%%%%%%%
\subsection{Sections}

\begin{frame}[fragile]{Using Sections}
\begin{itemize}[<+->]
	\item Treat sections just like you would in a paper.
	\item Use \verb=\tableofcontents[=$section$\verb=]= to keep audience informed of your talk's general plan.
	\item Use \verb=\AtBeginSection[]{=$specialframe$\verb=}= to help audience follow the structure of your talk.
\end{itemize}
\end{frame}

\begin{frame}[fragile,allowframebreaks=1.1]{Using Sections: Backstage}
\begin{semiverbatim}
	\\section\{Intro to Beamer\}
	\\subsection\{About Beamer\}
	\\subsection[Basic Structure]\{Basic Structure\}
	\\subsection\{How to Compile\}
	
	\\section\{Overlaying Concepts\}
	\\subsection\{Specifications\}
	\\subsection[Examples]\{Examples: Lists, Graphics, Tables\}
	
	\\section[Sparkle]\{Adding that Sparkle\}
	\\subsection\{Sections\}
	\\subsection\{Themes\}
	
	\\section*\{References\}
\end{semiverbatim}
\begin{semiverbatim}
\\AtBeginSection[]
\{
    \\begin\{frame\}\{Table of Contents\}
        \\tableofcontents[currentsection]
    \\end\{frame\}
\}
\end{semiverbatim}
\end{frame}

\begin{frame}{See, I can get a ToC anywhere!}
	\tableofcontents[currentsubsection]
\end{frame}

%%%%%%%%%%%%%%%%%%%%%%%%
\subsection{Themes}

\begin{frame}[fragile]{Pre-customized Templates}
\begin{itemize}
\item To make your presentation use a shiny theme like ours:
	\begin{itemize}
	\item Find \verb=\mode<presentation>{}= at the top of your file
	\item Insert \verb=\usetheme{Warsaw}= into the \verb={}=
	\end{itemize}
\item Also available within each Presentation Theme:
	\begin{itemize}
	\item \textbf{Color themes:} \verb=\usecolortheme{=\emph{colorthemename}\verb=}= \\ 
	  control colors for bullets, background, text, etc.
	\item \textbf{Inner themes:} \verb=\useinnertheme{=\emph{innerthemename}\verb=}= \\ 
	  control main title, environments, figures and tables, footnotes, etc.
	\item \textbf{Outer themes:} \verb=\useoutertheme{=\emph{outerthemename}\verb=}= \\ 
	  control head-/foot-lines, sidebars, frame titles, etc.
	\end{itemize}
\end{itemize}
\end{frame}

\begin{frame}[fragile]{A Sampling of Themes}
\begin{itemize}
\item \textbf{General themes:}
	\begin{tabular}{llll}
		default	&			Antibes			&	Berlin	&		Copenhagen \\
		Madrid	&			Montpelier	&	Ilmenau	&		Malmoe	\\
		CambridgeUS	&	Berkeley		&	Singapore	&	Warsaw	\\		
	\end{tabular}
\item Also available:
	\begin{itemize}
	\item \textbf{Color themes:} \\ beetle, beaver, orchid, whale, dolphin
	\item \textbf{Inner themes:} \\ circles, rectanges, rounded, inmargin
	\item \textbf{Outer themes:} \\ infolines, smoothbars, sidebar, split, tree
	\end{itemize}
\item See \htmladdnormallink{\underline{\textcolor{blue}{The Beamer	Theme Matrix}}}{http://www.hartwork.org/beamer-theme-matrix/}
\end{itemize}
\end{frame}

%%%%%%%%%%%%%%%%%%%%%%%%
%\subsection{Handouts}

%%%%%%%%%%%%%%%%%%%%%%%%%%%%%%%%%%%%%%%%
\section*{References}

\begin{frame}{Good sites to visit for Beamer help}
%\begin{scriptsize}
\begin{itemize}
	\item \htmladdnormallink{\underline{\textcolor{blue}{The Beamer	User Guide}}}
	{http://www.ctan.org/tex-archive/macros/latex/contrib/beamer/doc/beameruserguide.pdf}
	\item \htmladdnormallink{\underline{\textcolor{blue}{The Beamer Homepage}}}
	{http://latex-beamer.sourceforge.net/}		
	\item \htmladdnormallink{\underline{\textcolor{blue}{A Quick Tutorial}}}
	{http://heather.cs.ucdavis.edu/~matloff/beamer.html}
	\item \htmladdnormallink{\underline{\textcolor{blue}{A Beamer Quickstart}}}
	{http://www.math.umbc.edu/~rouben/beamer/}
	\item \htmladdnormallink{\underline{\textcolor{blue}{A Long Tutorial}}}
	{http://cs.guc.edu.eg/research/latex_online_tutorial/LaTeX/LaTeX_03/latex_beamer_01.html}
	\item \htmladdnormallink{\underline{\textcolor{blue}{\LaTeX\ + Beamer Examples}}}
	{http://www.rennes.enst-bretagne.fr/~gbertran/pages/tutorials_latex.html}
	\item \htmladdnormallink{\underline{\textcolor{blue}{A Beamer Presentation on Beamer}}}
	{http://www.uncg.edu/cmp/reu/presentations/Charles\%20Batts\%20-\%20Beamer\%20Tutorial.pdf}
\end{itemize}
%\end{scriptsize}
\end{frame}

\end{document}
