\chapter{Chapter Three}

%This chapter will  summarize the overall project, with a focus on lessons learned, implications for future research and intervention, and limitations. It can, but does not have to, include a more personal reflection on the research process. 

\section{Lessons Learned}
\begin{description}
	\item[Translation of messages may be best completed orally.] \hfill \\
	The process of translating all voice and text messages into the local dialect of Swahili proved to be much more challenging than expected. Since discussions with the CHVs and nurses during the Hear and Create phase were all conducted in English, it was difficult to translate into Swahili while maintaining the integrity of the original content. A woman local to the Ndivisi division was recruited to translate and later record the voice messages, as she was familiar with traditional Kenyan Swahili and Luhya, the language spoken by the native tribes in the region. However, the language spoken in Ndivisi division and surrounding areas blended elements of both languages; thus, it was difficult to come to a consensus on which translations would be understood by the majority of people living in the region. Additionally, many of the voice and text messages involved vocabulary and phrasing with which the translator was not familiar. Ultimately, the research team opted to use Google Translate as a tool for creating initial translations. These translations were then read to the translator, who then provided corrections orally. Although this process was extremely time-intensive, it yielded voice and text message prompts that were rated to be easy to understand and accurate by 100\% CHVs who participated in the usability survey at the conclusion of the study. 

	\item[Voice message quality is crucial to the success of an IVR system.] \hfill \\
	Prior to the mock testing session of the service during the Create phase, the research team recorded the complete set of audio messages at a residence in the nearby town of Webuye. At first, the team was not concerned about background noise from cows, chickens, or neighbors affecting the recording, as it would give the voice messages the feel of speaking to someone else in their home environment over the phone. However, testing of the voice messages during the mock session revealed that the audio messages suffered in quality when heard through users' phones. Numerous attempts were made to create quiet recording environments, free of animal or human sounds, but this proved to be impossible. Eventually, the research team discovered a recording studio in Eldoret town and opted to complete the recordings there. While studio-quality recordings came at a much higher price than originally intended, CHVs who responded to the usability survey believed that the voice messages were easy to understand and navigate. 

	\item[Expect mobile network variability.] \hfill \\
	Mobile network coverage in the study catchment area was extremely variable. At times, service would be at full strength, while at others, poor coverage made it impossible to complete calls. Given that the study site was in a very remote and rural region, this poor network coverage was not surprising. However, inconsistent mobile network signal did have an impact on results of the study, specifically when compiling data on usage of the system over time. A large percentage of calls initiated by CHVs were dropped in progress, either due to users hanging up early or due to poor network coverage. While user error may have accounted for some of the dropped calls, the quality of network coverage is likely to have caused problems for users trying to access the system. Unfortunately, there was no possible way for the research team to address this issue during the study.
\end{description}


\section{Limitations}
\paragraph{This study piloted an intervention that was previously untested in this target population. Thus, the scope of the study was limited to a single clinic and a small, convenience sample of CHVs working in the clinic's catchment area. Participants for focus groups were selected based on English comprehension and demonstrated interest in the study, and thus may not generally represent the perspectives or viewpoints of all other CHVs within the health system. Due to time contraints, only one cycle of the Hear and Create phases were completed; in future iterations, additional cycles of prototyping and mock testing would have been conducted in order to refine and create additional features for the system.  }

\section{Implications for Future Research}
\paragraph{Future studies can build upon these findings by expanding the scope of the study beyond one study clinic and its corresponding two community units. Additional focus groups from the CHV and nurse population would also contribute to a more robust exploration of the flow of information and people with respect to maternal and child healthcare at the community level in this region. From a design perspective, upcoming versions of the system will also aim to further integrate the Baby Monitor screening service with the patient management component, so as to alert CHVs and suggest steps for follow up when a woman presents with concerning screening results. This will enable CHVs to identify high-risk patients and to adjust their approach in visiting and caring for them at the community level. Future studies will also evaluate the impact of this patient management system on process outcomes such as number of home visits, clinic visits for prenatal care, and clinic deliveries. Future iterations of the intervention will also evaluate additional features that were suggested by CHVs during the mock testing sessions in the Create phase, such as text message reminders for CHVs about upcoming home visits and expected delivery dates of women enrolled with the Baby Monitor system from their village. }


\section{Conclusions}
\paragraph{Maternal and child mortality remains a public health problem in Kenya. Although many mHealth programs have implemented worldwide to improve maternal and child health care, few have leveraged both interactive voice response and text messaging modalities to engage with both patients and providers. This pilot study adopted a human-centered design framework to identify and address the challenges faced by community health volunteers in providing maternal and child health services. Based on these findings, a patient management system was created that allowed CHVs to report data and receive text message updates about patients in their communities. This system was generally well-received by the CHVs, as they used the system with regularity and provided high self-reported usability ratings.}
